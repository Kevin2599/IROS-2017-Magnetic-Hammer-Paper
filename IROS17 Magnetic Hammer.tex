

\documentclass[letterpaper, 10 pt, conference]{ieeeconf}  % Comment this line out if you need a4paper

%\documentclass[a4paper, 10pt, conference]{ieeeconf}      % Use this line for a4 paper

\IEEEoverridecommandlockouts                              % This command is only needed if 
                                                          % you want to use the \thanks command

\overrideIEEEmargins                                      % Needed to meet printer requirements.

% See the \addtolength command later in the file to balance the column lengths
% on the last page of the document

% The following packages can be found on http:\\www.ctan.org
%\usepackage{graphics} % for pdf, bitmapped graphics files
%\usepackage{epsfig} % for postscript graphics files
%\usepackage{mathptmx} % assumes new font selection scheme installed
%\usepackage{times} % assumes new font selection scheme installed
%\usepackage{amsmath} % assumes amsmath package installed
%\usepackage{amssymb}  % assumes amsmath package installed
\usepackage{cite}
\usepackage{mathtools}
\usepackage{cleveref}


\title{\LARGE \bf
Magnetic Hammer Actuation for Tissue Penetration using Millirobots
}


\author{Authors here% <-this % stops a space
\thanks{This work was supported by }% <-this % stops a space
\thanks{Authors are with the Dept. of Electrical and Computer
Engineering, University of Houston, Houston, TX 70004, USA
        {\tt\small jleclerc@uh.edu}}%
%\thanks{$^{2}$Bernard D. Researcheris with the Department of Electrical Engineering, Wright State University,
%       Dayton, OH 45435, USA
%      {\tt\small b.d.researcher@ieee.org}}%
}


\begin{document}



\maketitle
\thispagestyle{empty}
\pagestyle{empty}


%%%%%%%%%%%%%%%%%%%%%%%%%%%%%%%%%%%%%%%%%%%%%%%%%%%%%%%%%%%%%%%%%%%%%%%%%%%%%%%%
\begin{abstract}

Untethered navigation of millirobots within a human body using an MRI scanner is a promising technology for minimally invasive surgery or drug delivery. However, the magnetic field and gradient values are too small to produce forces large enough to penetrate tissues. This paper presents a method to produce large pulsed forces on millirobots. A magnetic sphere is placed inside the robot body and can move back and forth. This movement is created by alternatively changing the gradient direction. On the aft side, a spring allows the sphere to change direction smoothly. On the forward side, a hard rod creates a surface for the sphere to impact. This impact results in a large pulsed force. This system has first been modeled and simulated. Different control strategies are presented and experimentally tested. Finally, the system was tested to penetrate artificial tissues. 

\end{abstract}


%%%%%%%%%%%%%%%%%%%%%%%%%%%%%%%%%%%%%%%%%%%%%%%%%%%%%%%%%%%%%%%%%%%%%%%%%%%%%%%%
\section{INTRODUCTION}

The navigation of millimeter-scale robots \cite{ahu61} through the passageways of bodies is currently being studied as a method to perform highly localized drug delivery or perform minimally invasive surgery. Untethered navigation can be achieved by placing a magnetic piece inside the robot and producing a controlled magnetic field around a patient. Propulsion and steering of millirobots can be accomplished by either moving a permanent magnet assembly around a patient or by controlling the current inside electromagnets. The latest solution is often realized with an MRI scanner which already includes several electromagnets. In that case, the background field magnetizes the magnetic piece of the robot, and the gradients coils produce the magnetic gradient necessary to produce forces. The MRI scanner can be used at the same time to provide real-time imaging of the operating area as well as positioning of the robots.\par
The value of the forces generated on the millirobots is proportional to the field gradient strength. Commercial MRI scanner produces gradients in the range of 20 to 40 mT/m. This value could allow the navigation of robots through the body fluids but not the penetration of tissues. This paper presents a method that allows producing large pulsed forces for tissue penetration. This system has been called magnetic hammer.\par
The magnetic hammer is a system embedded into the millirobot. The millirobot has a tubular structure in which a ferromagnetic sphere can move back and forth. This movement is produced by alternatively changing the gradient direction. On the aft side of the millirobot, a spring allows the sphere to change direction smoothly. On the forward side, a hard rod creates a surface for the sphere to impact (impact plate). This impact results in large pulsed forces that allow penetrating body tissues progressively.
A magnetic test bench has been developed to make experimental tests more practical and less expensive. It includes coils, sensors, power electronics and a real-time controller.\par
The paper is organized as follow: first, the system is mathematically modeled, and its behavior is studied. Secondly, the parameters of the model are experimentally measured. Different materials for the impact plate are compared. Thirdly, the magnetic test bench is described, and different control methods are tested (open loop, partially closed loop, and closed loop). Results are compared to the mathematical model. Finally, the system is tested to penetrate artificial tissues made with agar. The last section is a conclusion of this study.


\section{Theoretical study}

\subsection{Mechanical modelization}

\subsection{Electromagnetic modelization}

The magnetic field generated by an MRI scanner can be separated into two components. The first is a constant, strong and uniform magnetic field B0. This field is used to align the magnetic moments of the protons. Commercial MRI scanners have a B0 ranging from 1.5 to 3 T. The second component of the field is the magnetic gradient. It is used to encode the MRI signal spatially. The flux density G produced by the gradient coils is superposed to B0 and linearly varies with position. A computer controls its value.
The modelisation of the field inside the uniformity sphere of an MRI scanner is straightforward. It is the sum of B0 and G. G is directly proportional to the current inside the gradient coils.

\begin{equation}
\mathbf{B}=\mathbf{B_0}+\mathbf{G},~~~
\mathbf{B_0}=\begin{bmatrix}
0\\ 
0\\ 
B_0
\end{bmatrix},~
\mathbf{G}=\begin{bmatrix}
k_x.I_x\\ 
k_y.I_y\\ 
k_z.I_z
\end{bmatrix}
\end{equation}

where $k_x$, $k_y$ and $k_z$ are the coils constants and $I_x$, $I_y$ and $I_z$ are the electrical current values.\par

The flux density is more complicated to calculate outside of the uniformity sphere. The same problem is present in our desktop experiment because the flux density and gradient are not constant. It is paramount to accurately compute the magnetic field to be able to calculate forces accurately. A method to calculate the field produced in all space by a solenoid assembly was used. It was tested on our desktop size experiment.
According to [], the magnetic flux density produced by a current loop in all space can be calculated using equations \cref{Bz1loop,Bteta1loop,delta,beta,single_loop_geometry}.
\begin{equation}
Bz=\frac{\mu _0.I}{2.\pi.\delta ^{2}.\beta  }\left [ \left ( a^2-\rho ^2-z^2 \right )(E(k^2)+\delta ^2.K(k^2)) \right ] 
\label{Bz1loop}
\end{equation}
\begin{equation}
Bz=\frac{\mu _0.I.z}{2.\pi.\delta ^{2}.\beta.\rho   }\left [ \left ( a^2-\rho ^2-z^2 \right )(E(k^2)-\delta ^2.K(k^2)) \right ]
\label{Bteta1loop}
\end{equation}
\begin{equation}
\delta =\sqrt{a^2+R_m^2+Z_m^2-2.a.R_m}
\label{delta}
\end{equation}
\begin{equation}
\beta =\sqrt{a^2+R_m^2+Z_m^2+2.a.R_m}
\label{beta}
\end{equation}

\begin{figure}
  \includegraphics[width=\linewidth]{single_loop.png}
  \caption{Geometry and variables used in equation \cref{Bz1loop,Bteta1loop,delta,beta}}
  \label{single_loop_geometry}
\end{figure}

The cross-section S of any solenoid can be divided into infinitesimal sections dS. Each dS is subjected to a current dI=J.dS. This current dI forms an infinitesimal loop, and the field it produces can be calculated using equation 1. By integrating this equation over the solenoid cross-section, one can obtain the value of the flux density generated by the solenoid.\par
The flux density has to be calculated for each solenoid. The total flux density is the vectorial sum of the flux density produced by each solenoid.


\subsection{Simulation results}


\section{Experimental determination of model parameters}

\subsection{Impact coefficient of resititution}

\subsection{Frictional coefficient}

\section{Experimental magnetic hammer tests}

\subsection{Magnetic test bench description}

\subsection{Open-loop experiment}

\subsection{Partially closed loop experiment}

\subsection{Closed loop experiment}

\section{Tissue penetration test}
%\begin{table}[h]
%\caption{An Example of a Table}
%\label{table_example}
%\begin{center}
%\begin{tabular}{|c||c|}
%\hline
%One & Two\\
%\hline
%Three & Four\\
%\hline
%\end{tabular}
%\end{center}
%\end{table}


%   \begin{figure}[thpb]
%      \centering
%      \framebox{\parbox{3in}{We suggest that you use a text box to insert a graphic (which is ideally a 300 dpi TIFF or EPS file, with all fonts embedded) because, in an document, this method is somewhat more stable than directly inserting a picture.
%}}
%      %\includegraphics[scale=1.0]{figurefile}
%      \caption{Inductance of oscillation winding on amorphous
 %      magnetic core versus DC bias magnetic field}
 %     \label{figurelabel}
 %  \end{figure}
   



\section{CONCLUSIONS}


\addtolength{\textheight}{-12cm}   % This command serves to balance the column lengths
                                  % on the last page of the document manually. It shortens
                                  % the textheight of the last page by a suitable amount.
                                  % This command does not take effect until the next page
                                  % so it should come on the page before the last. Make
                                  % sure that you do not shorten the textheight too much.

%%%%%%%%%%%%%%%%%%%%%%%%%%%%%%%%%%%%%%%%%%%%%%%%%%%%%%%%%%%%%%%%%%%%%%%%%%%%%%%%



%%%%%%%%%%%%%%%%%%%%%%%%%%%%%%%%%%%%%%%%%%%%%%%%%%%%%%%%%%%%%%%%%%%%%%%%%%%%%%%%


\bibliography{biblio}



\end{document}
